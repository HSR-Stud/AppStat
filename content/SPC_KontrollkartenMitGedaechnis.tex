\subsection{Kontrollkarten mit Gedächnis}
\begin{itemize}
	\item Bei klassischen (Shewart) Kontrollkarten erfolgt Entscheid über Eingriff auf Basis des Ergebnisses einer Stichprobe
	\begin{itemize}
		\item Vergangenheit wird nicht mitberücksichtigt
	\end{itemize}
	\item Wird die Vergangenheit mit Berücksichtigt ist es eine \textbf{Kontrollkarte mit Gedächtnis}
	\item Vorteil: Störungen können schneller aufgedeckt werden
	\item Kontrollkarten mit Gedächtnis basieren auf Linearkombination einiger (oder aller)  vor Zeitpunkt $i$vorgenommenen Stichproben
	\item Die allgemeine Formel dafür ist Formel \ref{eq:gedaechtnisAlg}
	\item Die Parameter $\alpha_i$ und $\beta_i$ legen den Typ der Kontrollkarte fest
	\begin{itemize}
		\item Wenn z.B $\alpha_i = 0$ , $\beta_1\ldots\beta_{i-1} = 0$ und $\beta_i=1$ ist das die gedächtnislose Shewart $\bar{x}$-Karte
	\end{itemize}
	\item Je kleiner $\left\vert \delta\right\rvert$ ist (siehe Formel \ref{eq:delta}), desto rascher wird eine Abweichung vom Zielwert ($\mu_0$) erkannt.
	\begin{itemize}
		\item Desto kürzer ist die ARL (mittlere Lauflänge, Kapitel \ref{subsubsec:MittlereLauflaenge})
	\end{itemize}
\end{itemize}
\begin{align}
	\label{eq:gedaechtnisAlg}
	y_i&= \alpha_i + \sum_{j=1}^{i}\beta_j\bar{x}_i
\end{align}

\subsubsection{CUMSUM}
\begin{itemize}
	\item Cumulative Sum Control Chart, Karte zur Kontrolle der Mittelwerte
	\item Trägt die Kummulative Abweichung der Stichproben zu dem Zielwert auf, sofern diese eine gewisse Grenze überschreiten
	\item Kann sowohl für einzlene Beobachtungen, wie auch für Stichproben verwendet werden
	\item Für Prozesse Empfohlen bei denen bereits kleine Störungen bedeutsam sind und aufgedeckt werden müssen
	\item CUMSUM-Karte verwendet zwei \glqq Speicherwerte\grqq für Aufsummieren der Abweichung
	\begin{itemize}
		\item Formel \ref{eq:CUMSUMPlus} für obere Grenze und Formel \ref{eq:CUMSUMPlus} für untere Grenze
		\item Beide dieser Werte müssen mit jedem neuen Messwert aktualisiert werden
		\item $K$ ist hierbei der Referenzwert, es werden nur Fehler erkannt, die diesen Überwiegen
		\item Ein guter Wert für $K$: siehe Formel \ref{eq:CUMSUMK}
		\item Soll ein Shift von $\Delta$ aufgedeckt werden $\Rightarrow K =\frac{\Delta}{2}$ 
		\item Es werden also nur Werte Berücksichtigt die einen Abstand $>K$ von $\mu_0$ haben
	\end{itemize}
	\item Der Prozess ist unter Kontrolle, wenn der Erwartungswert von $E(C_i^+)=E(C_i^-)=0$ ist
	\item Der Prozess ist ab einem gewissen Index $i_0$ ausser Kontrolle wenn einer Erwartungswert eine Konstante $H$ überschreitet
	\begin{itemize}
		\item Den Prozess stoppen und in Ordnung bringen, Anschliessend neustarten und $C_i^+ = C_i^- = 0$ setzten
		\item $H$ wird auch \textbf{Entscheidungsintervall} genannt
		\item Ein guter Wert für $H$ ist in Formel \ref{eq:CUMSUMK} gezeigt (wenn zulässiger Bereich $\pm 3\sigma$ ist)
	\end{itemize}
\end{itemize}
\begin{align}
	\label{eq:CUMSUMPlus}
	C_i^+&=\max\left\lbrace 0,\bar{x}_i-(\mu_0+K)+C_{i-1}^+\right\rbrace\\
	\label{eq:CUMSUMMinus}
	C_i^-&=\max\left\lbrace 0,(\mu_0-K)-\bar{x}_i+C_{i-1}^-\right\rbrace\\
	\label{eq:CUMSUMK}
	K &= \frac{\hat{\sigma}}{2}\qquad \text{und}\qquad H=5\hat{\sigma}
\end{align}

\subsubsection{EMWA}
\begin{itemize}
	\item \textbf{Exponentially Weighted Movin Average} Karte zur Mittelwertskontrolle
	\item Gewicht $\beta$ in Formel \ref{eq:gedaechtnisAlg} klingt expontentiell ab (mit dem zuhnemenden \glqq Alter \grqq der jeweiligen Stichprobe)
	\item Der Glättungsfaktor $\lambda$ bestimmt den Einfluss der zurückliegenden Stichprobenmittelwerte $\bar{x}_j$
	\begin{itemize}
		\item Je kleiner $\lambda$ dest mehr Werte $\bar{x}_j$ werden zur Entscheidung herangezogen (langes Gedächtnis)
		\item Die Grenzen UCL$_i$ und LCL$_i$ steigen bei grösserem $\lambda$ schneller an
		\item EMWA-Karte hat somit ein ungleichmässiges, unbegrenztes Gedächtnis
		\item Für $\lambda=1$ haben wir die Shewhart $\bar{x}$-Karte
	\end{itemize}
	\item Den Wert $y_i$ zum $i.$ Zeitpunkt kann gemäss Formel \ref{eq:EWMAalg} absolut berechnet werden oder mit Formel \ref{eq:EWMArek} rekursiv
	\item Die Kontrollgrenzen berechnen sich gemäss Formel \ref{eq:EWMAkontrollgrenze}
	\begin{itemize}
		\item \textbf{Achtung}: Diese ist Abhängig von der Messnummer $i$ und ändert sich somit mit jeder Messung 
		\item Änderung zu beginn stark, anschliessend immer weniger, Funktion sieht aus wie Schrittantwort von PT1-Glied
		\item Bereits für Mittelgrosse $i$ ist die Funktion asymptotisch und es ergeben sich die von $i$ unabhängigen Grenezn in Formel \ref{eq:EWMAkontrollgrenzeinf}
	\end{itemize}
\end{itemize}
\begin{align}
	\label{eq:EWMAalg}
	y_i &= \underbrace{(1-\lambda)^{i}\mu_0}_{\alpha_i}+\sum_{j=1}^{i}\underbrace{(1-\lambda^{i-j})}_{\beta_i}\bar{x}_j\\
	\label{eq:EWMArek}
	y_i&=(1-\lambda)y_{i-1}+\lambda\bar{x}_i\\
	\label{eq:EWMAkontrollgrenze}
	\text{UCL}_i &= \mu_0 + 3\frac{\sigma}{\sqrt{n}}\sqrt{\frac{\lambda}{2-\lambda}(1-(1-\lambda)^{2i})}	\quad &\text{und} \qquad
	\text{LCL}_i &= \mu_0 - 3\frac{\sigma}{\sqrt{n}}\sqrt{\frac{\lambda}{2-\lambda}(1-(1-\lambda)^{2i})}\\
	\label{eq:EWMAkontrollgrenzeinf}
	\text{UCL} &= \mu_0 + 3\frac{\sigma}{\sqrt{n}}\sqrt{\frac{\lambda}{2-\lambda}} 	\quad &\text{und} \qquad
	\text{LCL} &= \mu_0 - 3\frac{\sigma}{\sqrt{n}}\sqrt{\frac{\lambda}{2-\lambda}}
\end{align}
